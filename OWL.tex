\documentclass{beamer}
\usepackage{listings}
\usepackage[utf8x]{inputenc}
\usepackage{color}
\lstset{
  language=xml,
  float=htbp,
        tabsize=4,
        linewidth=\linewidth,
        breaklines=true,
        breakatwhitespace=true,
        basicstyle=\scriptsize\ttfamily,
        numberfirstline=false,
        numberstyle=\scriptsize,
        stepnumber=1,
        numbersep=5pt,
        showspaces=false,
        showtabs=false,
        showstringspaces=false,
        showlines=false,
        extendedchars=true,
        identifierstyle=\bfseries,
        keywordstyle=\bfseries,
        commentstyle=\itshape,
        stringstyle=\ttfamily,
        flexiblecolumns=false,
        fontadjust=true,
        captionpos=b,
}

\usetheme{Boadilla}
\definecolor{specialblue}{rgb}{0.2,0.2,0.7}

\title[OWL]{Ontologien und formale Semantik in OWL}
\author{Timon Link, Matthias Jurisch}
\date{4. Dezember 2012}
\begin{document}
\begin{frame}
\titlepage
\end{frame}

\begin{frame}{Übersicht}
\tableofcontents
\end{frame}

%%%%%%%%%%%%%%%%%%%%%% Grundlagen %%%%%%%%%%%%%%%%%%%%%%%%%%
%%%%%%%%%%%%%%%%%%%%%% T W3C-Scheiss %%%%%%%%%%%%%%%%%%%%%%%%%%
% Working group: http://www.w3.org/2001/sw/WebOnt/
% OWL bwcomes recommendation: http://www.w3.org/2001/sw/WebOnt/
% OWL official recommendation: http://www.w3.org/News/2004.html
\section{Historie und Herkunft}
\begin{frame}{Historie und Herkunft}
	\begin{itemize}
	\item OWL - Web Ontology Language (Warum nicht WOL?)
	\item Name entstanden aus 3 Gründen:
		\begin{itemize}
			\item es ist klar, wie er ausgesprochen wird (engl. Eule) 
			\item er eröffnet gute Möglichkeiten für Logos
			\item Eulen werden mit Weisheit in Verbindung gebracht
		\end{itemize}
	\item 01. November 2001 Web-Ontologie Working Group wurde gegründet
	\item 15. Dezember 2003 OWL als Vorschlag bei W3C vorgestellt
	\item 10. Februar 2004 wird OWL zum offizielen \textit{recommendation}
	\item 30. Mai 2004 wurde die Web-Ontologie Working Group geschlossen
	\item 6. September 2007 Gründung der OWL Working Group 
	\end{itemize}
\end{frame}
%%%%%%%%%%%%%%%%%%%%%% T Rückblick RDF %%%%%%%%%%%%%%%%%%%%%%%

\section{Kurzer Rückblick auf RDF}
\begin{frame}{RDF - ein kurzer Rückblick}
\begin{itemize}
	\item \alert{R}esource \alert{D}escription \alert{L}anguage
	\item formale Sprache zur Beschreibungen einfacher Ontologien
	\item Graphen statt Bäume (trotz XML Syntax)
	\item Beschreibung eines Graphen mit Tripeln: (subject, predicat, object)
	\item Was hat RDF mit OWL zu tun?
\end{itemize}
\begin{block}{Genau das:}
	RDF(S) ist in OWL (Full) enthalten\\
	Daher ist jedes RDF(S) Dokument auch ein OWL Dokument.
\end{block}
\end{frame}

%%%%%%%%%%%%%%%%%%%%%% M Ausdruck von RDF-Konstrukten einfacher mit OWL (pointer auf Header auch)

\section{Vereinfachung von RDF-Ausdrücken mit OWL}
\begin{frame}[fragile]{Vereinfachung von RDF-Ausdrücken mit
OWL}
\begin{itemize}
\item RDF/RDFS lässt sich durch OWL einfacher und abgekürzt ausdrücken
\end{itemize}

\begin{block}{Daher}
\begin{itemize}
\item Kurze Auflistung von Abkürzungen.
\item kennenlernen der wesentlichen Sprachkonstrukte
\end{itemize}
\end{block}


\end{frame}
\begin{frame}[fragile]{Vereinfachung von RDF-Ausdrücken mit
OWL}{Klassen}
\begin{block}{Klassen in RDFS}
\begin{lstlisting}[lang="xml"]
<rdf:Description about="&ex;Klassenname">
  <rdf:type resource="&rdfs;Class"/>
</rdf:Description>
\end{lstlisting}
\end{block}
\begin{block}{OWL}
\begin{lstlisting}[lang="xml"]
<owl:Class about="&ex;Klassenname"/>
\end{lstlisting}
\end{block}

\end{frame}

\begin{frame}[fragile]{Vereinfachung von RDF-Ausdrücken mit
OWL}{Individuen}
\begin{block}{Individuen in RDFS}
\begin{lstlisting}[lang="xml"]
<rdf:Description about="&ex;individuum">
  <rdf:type resource="&ex;Klassenname"/>
</rdf:Description>
\end{lstlisting}
\end{block}
\begin{block}{OWL}
\begin{lstlisting}[lang="xml"]
<ex:individuum about="&ex;Klassenname"/>
\end{lstlisting}
\end{block}
\end{frame}

\begin{frame}[fragile]{Vereinfachung von RDF-Ausdrücken mit
OWL}{Unterklassen}
\begin{block}{Unterklassen in RDFS}
\begin{lstlisting}[lang="xml"]
<rdf:Description about="&ex;Klassenname">
  <rdf:type resource="&rdfs;Class"/>
  <rdfs:subClassOf rdf:resource="&ex;Oberklasse"/>
</rdf:Description>
\end{lstlisting}
\end{block}
\begin{block}{OWL}
\begin{lstlisting}[lang="xml"]
<owl:Class about="&ex;Klassenname">
  <rdfs:subClassOf rdf:resource="&ex;Oberklasse"/>
</owl:Class>
\end{lstlisting}
\end{block}

\end{frame}

\begin{frame}[fragile]{Vereinfachung von RDF-Ausdrücken mit OWL}{Properties}
\begin{itemize}
\item RDF-Propertys $\rightarrow$ OWL-Rollen
\begin{block}{Abstrakte Rollen}
\begin{itemize}
\item Verbindungen zwischen Individuen und Individuen
\item Knotentyp: \tt owl:ObjectProperty
\end{itemize}
\end{block}

\begin{block}{Konkrete Rollen}
\begin{itemize}
\item Verbindungen zwischen Individuen und Datenwerte
\item Knotentyp: \tt owl:DatatypeProperty
\end{itemize}
\end{block}

\item Beide sind Unterklasse von \texttt{rdf:Property}
\item Verwendung dementsprechend analog zu \texttt{rdf:Property}

\end{itemize}
\end{frame}


%%%%%%%%%%%%%%%%%%%%%% M was kann nicht in RDF ausgedrückt werden?

\section{Grenzen von RDF}

\begin{frame}[fragile]{Grenzen von RDF}
\begin{itemize}
\item Negative Aussagen in RDF
\begin{itemize}
\item Nicht Möglich
\item Auch nicht über Umwege
\end{itemize}
\end{itemize}
\begin{exampleblock}{Beispiel}
\begin{lstlisting}
ex:sebastian rdf:type ex:Nichtraucher.
ex:sebastian rdf:type ex:Raucher.
\end{lstlisting}
$\rightarrow$ Kein Widerspruch, auch nicht Möglich diesen auszudrücken
\end{exampleblock}

\begin{itemize}
\item OWL:
\begin{itemize}
\item erlaubt solche Ausdrücke
\item übersteigt damit die Mächtigkeit von RDF
\end{itemize}
\end{itemize}
\end{frame}

%%%%%%%%%%%%%%%%%%%%%%-Inhalt-
%%%%%%%%%%%%%%%%%%%%%% -> Drei OWL-Varianten
%%%%%%%%%%%%%%%%%%%%%% T eingehena auf Tabelle 

\section{OWL-Varianten}
\begin{frame}{OWL - Teilsprachen}
\begin{itemize}
	\item OWL ist durch Mächtigkeit sehr komplex
	\item Deshalb Aufteilung in drei Teilsprachen:
\begin{itemize}
	\item OWL Full
	\item OWL DL
	\item OWL LITE
\end{itemize}
	\item unterschiedliche Komplexität und Mächtigkeiten
\end{itemize}
\begin{block}{Relationen}
	Es gilt:\\
	$OWL Lite \subseteq OWL DL \subseteq OWL Full$
\end{block}
\end{frame}

%
\begin{frame}{OWL Full}
\begin{itemize}
	\item enthält als einzige Teilsprache \alert{ganz} RDFS
	\item enthält OWL DL und OWL LITE als Teilsprachen
	\item sehr ausdrucksstark
	\item ABER: \alert{unentscheidbar}
	\item deshalb keine Praxisrelevanz
\end{itemize}
\end{frame}

%
\begin{frame}{OWL DL}
\begin{itemize}
	\item enthält OWL Lite und ist Teilsprache von OWL Full
	\item enscheidbar
	\item sehr Praxisrelevant
	\item daher wird es von den meisten Softwarewerkzeugen unterstützt
\end{itemize}
\begin{block}{Komplexität}
	NExpTime
\end{block}
\end{frame}

%
\begin{frame}{OWL Lite}
\begin{itemize}
	\item ist Teilsprache von OWL DL und OWL Full
	\item entscheidbar
	\item weniger ausdrucksstark
	\item deswegen kaum Praxisrelevanz
\end{itemize}
\begin{block}{Komplexität}
	ExpTime
\end{block}
\end{frame}


%%%%%%%%%%%%%%%%%%%%%% T Neues in OWL, dass Ausdrucksmöglichkeiten von 
%RDF übersteigt (Hier Header beschreiben)
\section{Neue Ausdrucksmöglichkeiten in OWL}
\begin{frame}{Neuerungen in OWL}
In OWL gibt es neue Sprachkonstrukte:
\begin{itemize}
	\item Der Kopf einer OWL-Ontologie
	\item Beziehungen zwischen Individuen
	\item Klassen Beziehungen
	\item Abgeschlossene Klassen
	\item Logische Konstruktoren
	\item Rolleneigenschaften
\end{itemize}
Im Folgenden werden diese Sprachkonstrukte genauer behandelt.
\end{frame}

\begin{frame}[fragile]{Kopf einer OWL-Ontologie}
\begin{block}{Kopf einer OWL-Ontologie, Aufgabe:}
Speicherung von Informationen, die für das gesamte Dokument gültig sind.
\end{block}
\begin{itemize}
	\item Deklarationen von Namensräumen analog zu RDF(s)
	\item weitere Informationen werden in diesem  Tag definiert: 
\end{itemize}
\begin{block}{OWL Ontology}
\begin{lstlisting}[lang="xml"]
<owl:Ontology rdf:about=""> ... <owl:Ontology>
\end{lstlisting}
\end{block}
\end{frame}

\begin{frame}{Kopf einer OWL-Ontologie II}
Inhalte des OWL Headers:
\begin{itemize}
	\item bekannte Strukturen aus RDFS:
	\begin{itemize}
		\item  \texttt{rdfs:comment} - Kommentare zur Ontologie
		\item  \texttt{rdfs:label} - Beschreibung der Ontologie
		\item  \texttt{rdfs:seeAlso} - Externe Verweise
		\item  \texttt{rdfs:isDefinedBy} - Subjekt "wird definiert von" Objekt
	\end{itemize}
\end{itemize}
\begin{itemize}
	\item Neue Strukturen aus OWL:
	\begin{itemize}
		\item \texttt{owl:versionInfo} - Versionsangabe der Ontologie
		\item \texttt{owl:priorVersion} - Vorgängerversion
		\item \texttt{owl:backwardCompatibleWith} - Abwärtskompatibilität
		\item \texttt{owl:incompatibleWith} - Inkompatibel
		\item \texttt{owl:DeprecatedClass} - veraltete Klasse
		\item \texttt{owl:DeprecatedProperty} - veraltete Property
	\end{itemize}
\end{itemize}
\end{frame}

\begin{frame}[fragile]{Kopf einer Ontologie Beispiel}
\begin{block}{Beispiel}
\begin{lstlisting}[lang="xml"]
<owl:Ontology rdf:about=""> 
  <rdfs:comment rdf:datatype="string">
    Ontologie der Kochplattform 
    Rezepte zum Ausprobieren.		
  </rdfs:comment>
  <rdfs:label rdf:datatype="string">
    Rezepte, Kochen, Backen
  </rdfs:label>
  <owl:versionInfo>1.0.4</owl:versionInfo>
  <owl:DeprecatedClass rdf:ID="Suppe">
    <rdfs:comment  rdf:datatype="string">
      Eintopf ersetzt Suppe
    </rdfs:comment>
  </owl:DeprecatedClass>
<owl:Ontology>
\end{lstlisting}
\end{block}
\end{frame}


%%%%%%%%%%%%%%%%%%%%%% M einige Sprachkonstrukte 

\begin{frame}[fragile]{Beziehungen zwischen Individuen}
\begin{itemize}
\item In OWL lassen sich beziehungen zwischen Individuen ausdrücken
\item Es existiert keine \emph{Unique Name Assumption}
\item Mit \texttt{owl:sameAs} lässt sich ausdrücken, dass zwei
Individuen identisch sind.

\begin{exampleblock}{Beispiel}
\begin{lstlisting}[lang="xml"]
<Suppe rdf:about="Kohlsuppe"/>
<rdf:Description rdf:about="OmasKohlSuppe">
  <owl:sameAs rdf:resource="Kohlsuppe"/>
</rdf:Description>
\end{lstlisting}
\end{exampleblock}
\item \alert{Schlussfolgerung:} Im Beispiel ist \texttt{OmasKohlSuppe} auch eine
\texttt{Suppe}
\end{itemize}
\end{frame}

\begin{frame}[fragile]{Beziehungen zwischen Individuen}
\begin{itemize}
\item Ähnliches ist auch umgekehrt möglich
\item \texttt{owl:differentFrom} gibt dies an
\end{itemize}
\begin{block}{Abkürzende Schreibweise für viele Individuen}
\begin{lstlisting}[lang="xml"]
<owl:AllDifferent>
  <owl:distinctMembers rdf:parseType="Collection">
    <Suppe rdf:about="Gulaschsuppe"/>
    <Suppe rdf:about="UngarischeGulaschsuppe"/>
    <Suppe rdf:about="Kohlsuppe"/>
  </owl:distinctMembers>
</owl:Alldifferent>
\end{lstlisting}
\end{block}

\end{frame}
%
\begin{frame}[fragile]{Klassenbeziehungen}
\begin{itemize}
\item Äquivalente Relationen zu \texttt{owl:sameAs} und \texttt{owl:differentFrom}
existieren auch für Klassen
\item \texttt{owl:disjointWith}
\begin{itemize}
 \item disjunktheit zweier Klassen
\end{itemize}
\item \texttt{owl:equivalentClass}
\begin{itemize}
 \item Gleichheit zweier Klassen
\end{itemize}
\end{itemize}
\end{frame}
%
\begin{frame}[fragile]{Abgeschlossene Klassen}
Dieses Konstrukt beschreibt, dass eine Klasse \emph{vollständig} ist. D.h. keine weiteren Individuen existieren		
\begin{block}{Beispiel}
\begin{lstlisting}[lang="xml"]
<owl:Class rdf:about="GemischtesHackFleich">
  <owl:oneOf rdf:parseType="Colection">
    <Fleisch rdf:about="Schwein" />
    <Fleisch rdf:about="Rind" />
  </owl:oneOf>
</owl:Class>
\end{lstlisting}
\end{block}
Das Beispiel sagt aus, dass \texttt{Schwein} und \texttt{Rind} die einzigen Individuen der Klasse \texttt{GemischtesHackFleisch} sind.
\end{frame}

\begin{frame}[fragile]{Abgeschlossene Klassen}
Will man noch Beschreiben, dass ein bestimmtes Individuum nicht zu einer Klasse
gehört, muss folgendes ergänzt werden:	
\begin{block}{Beispiel}
\begin{lstlisting}[lang="xml"]
<Fleisch rdf:about="Huehnchen" />
<owl:allDifferent rdf:parseType="collection">
  <Fleisch rdf:about="Schwein" />
  <Fleisch rdf:about="Rind" />
  <Fleisch rdf:about="Huehnchen" />
</owl:allDifferent>
\end{lstlisting}
\end{block}
Daraus folgt, dass \texttt{Huehnchen} \alert{kein} Individuum der Klasse 
\texttt{GemischtesHackFleisch} ist,
da mittels \texttt{owl:allDifferent} bestimmt wird, dass Hühnchen \alert{nicht} mit \texttt {Schwein}
und \texttt{Rind} identisch sein kann.
\end{frame}
%
\begin{frame}[fragile]{Logische Konstruktoren}
\begin{itemize}
\item \emph{Disjunktion, Konjunktion} und \emph{Negation} können
ausgedrückt werden.
\item In \emph{OWL Lite} nur beschränkt erlaubt.
\item Werden als \emph{Logische Konstruktoren} bezeichnet
\item Auf diese Weise können komplexe Klassen erstellt werden
\item Sprachkonstrukte:
\begin{itemize}
\item \texttt{owl:intersectionOf}
\item \texttt{owl:unionOf}
\item \texttt{owl:complementOf}
\end{itemize}
\item Schachtelung Möglich
\end{itemize}
\end{frame}
%
\begin{frame}[fragile]{Logische Konstruktoren}
\begin{exampleblock}{Beispiel}
\begin{lstlisting}[lang="xml"]
<owl:Class rdf:About="Eintopf>
  <rdfs:subClassOf>
    <owl:unionOf rdf:parseType="Collection">
      <owl:intersectionOf rdf:parseType="Collection">
        <owl:Class rdf:about="Suppen"/>
        <owl:Class rdf:about="MachtSatt"/>
      </owl:intersectionOf>
      <owl:complementOf>
        <owl:Class rdf:about="KlareSuppen"/>
      </owl:complementOf>
    </owl:unionOf>
  </rdfs:subClassOf>
</owl:Class>
\end{lstlisting}
\end{exampleblock}

\end{frame}
%

\begin{frame}[fragile]{Rolleneigenschaften}{Restrictions}
\begin{itemize}
\item Rollen können eingeschränkt werden
\item Beschränkung ist von der Klasse abhängig
\item \texttt{owl:Restriction} beschreibt Klassen
\item \texttt{owl:onProperty} ist die Rolle, auf die sich bezogen wird
\item Mögliche Einschränkungen:
\begin{itemize}
\item Werte:
\begin{itemize}
\item \texttt{owl:allValuesFrom}
\item \texttt{owl:someValuesFrom}
\item \texttt{owl:hasValue}
\end{itemize}
\item Kardinalität:
\begin{itemize}
\item \texttt{owl:maxCardinality}
\item \texttt{owl:minCardinality}
\item \texttt{owl:Cardinality}
\end{itemize}

\end{itemize}

\end{itemize}


\end{frame}
%

\begin{frame}[fragile]{Rolleneigenschaften}{Restrictions}
\begin{exampleblock}{Beispiel}
\begin{lstlisting}[lang="xml"]
<owl:Class rdf:about="Gemueseeintopf">
  <rdfs:subClassOf>
    <owl:Restriction>
      <owl:onProperty rdf:resource="hatZutat"/>
      <owl:minCardinality rdf:datatype="&xsd;nonNegativeInteger">
        3
      </owl:minCardinality>
      <owl:someValuesFrom rdf:resource="Gemuese"/>
    </owl:Restriction>
  </rdfs:subClassOf>
</owl:Class>
\end{lstlisting}
\alert{Nicht:} \emph{,,Ein Gemüseeintopf hat mindestens drei
Zutaten vom Typ Gemüse.''}
\end{exampleblock}
\end{frame}


%
\begin{frame}[fragile]{Rolleneigenschaften}{Beziehungen}
\begin{itemize}
\item Spezifikation von sogenannten \emph{characteristics}
\item Abgebildet über \texttt{rdf:type}
\item Mögliche Characteristics:
\begin{description}
\item[\texttt{owl:TransitiveProperty}] transitivität einer Rolle
\item[\texttt{owl:SymmetricProperty}] wechselseitige Beziehungen
\item[\texttt{owl:FunctionalProperty}] eindeutige Beziehungen
\item[\texttt{owl:InverseFunctionalProperty}] umgekehrt eindeutige Beziehungen
\item[\texttt{owl:inverseOf}] Property ist die Inverse einer anderen 


\end{description}
\item Weitere Details bei formaler Semantik
\item Anwendung wie bei RDF-Propertys
\end{itemize}
\end{frame}
%%%%%%%%%%%%%%%%%%%%%% formale Semantik von OWL

\section{Formale Semantik von OWL}

\begin{frame}[fragile]{Formale Semantik von OWL}{Die Beschreibungslogik ALC}
ALC besteht aus Klassen, Rollen und Individuen. OWL DL lässt sich mittels ALC vollständig
beschreiben.
\begin{block}{ALC Ausdrücke}
\begin{itemize}
	\item \texttt{Eintopf(ChiliConCarne)} \texttt{ChiliConCarne} $ \rightarrow $ ist Individuum von \texttt{Eintopf} 
	\item \texttt{IstZutat(GemischtesHackFleisch, ChiliConCarne)} $ \rightarrow $ Rollenbeziehung
	\item \texttt{Eintopf $ \subseteq $ FluessigGericht} $ \rightarrow $ Unterklassenbeziehung
	\item \texttt{Suppe $ \equiv $ Eintopf} $ \rightarrow $ Äquivalenz
	\item \texttt{$ \sqcap, \sqcup, \neg $} $ \rightarrow $ Konjunktion, Disjunktion, Negation
\end{itemize}
\texttt{ChiliConCarne $ \subseteq $ (FleischGericht $ \sqcap $ FluessigGericht) $ \sqcup $ (FleischGericht $ \sqcap \neg $Auflauf))}
\end{block}
\end{frame}

\begin{frame}{Formale Semantik von OWL}{Die Beschreibungslogik ALC II}
Komplexe Klassen können mit Hilfe von Quantoren bexchrieben werden.
\begin{block}{Quantoren}
	Ist R eine Rolle und C eine Klasse, so gilt: \texttt{$\forall $R.C} und \texttt{$ \exists $R.C}\\
	Beispiel:\\
	\texttt{GemischtesHackFleisch $ \subseteq $ $ \forall $hatZutat.Fleisch}\\
	$ \rightarrow $ Alle Zutaten von GemischtesHackFleisch müssen vom Typ Fleisch sein.
	\texttt{Fleischgericht $ \subseteq $ $ \exists $hatZutat.Fleisch}\\
	$ \rightarrow $ Ein Fleischgericht hat mindestens eine Fleischsorte als Zutat.
\end{block}
\end{frame}

\begin{frame}{Formale Semantik von OWL}{Modelierung von OWL DL mittels ALC}
\begin{block}{modellierung von OWL DL Konstrukten in ALC}
\begin{itemize}
	\item \texttt{owl:Nothing} $ \rightarrow $ $ \perp \equiv C \sqcap \neg C $
	\item \texttt{owl:Thing} $ \rightarrow $ $ \top \equiv C \sqcup \neg C $
	\item \texttt{owl:disjointWith} $\rightarrow $ $ C \sqcap D \equiv \top  $	
	\item \texttt{rdfs:range} $\rightarrow $ $ \top \subseteq \forall R.C $ (Range von R ist C)
	\item \texttt{rdfs:domain} $\rightarrow $ $ \exists R.\top \subseteq C $ (Domain von R ist C)
\end{itemize}
\end{block}
\end{frame}

\section{Automatisiertes Schlussfolgern mit OWL}
\begin{frame}{Automatisiertes Schlussfolgern mit OWL}{Das Tableauverfahren}
Wir nennen die Menge aus ALC Ausdrücken einer Ontologie \emph{Wissensbasis W}.
Transformation von ALC in \emph{Negationsnormalform}:
\begin{itemize}
	\item ersetze alle $ C \equiv D $ durch $ C \subseteq D $ und $ D \subseteq C $
	\item ersetze alle $ C \subseteq D $ durch $ \neg C \sqcup D $
\end{itemize}
\begin{block}{Wissensbasis}
	Resultierende Wissensbasis: \emph{NNF(W)}\\
	$ \rightarrow $ NNF(W) und W sind \emph{logisch equivalent}
\end{block}
Das Tableauverfahren wertet aus, ob eine Wissensbasis erfüllbarbar ist. 
\end{frame}

\begin{frame}{Automatisiertes Schlussfolgern mit OWL}{Das Tableauverfahren}
Vorgehen beim Tableauverfahren:
\begin{itemize}
	\item Überführe Wissensbasis W in NNF(W)
	\item Löse logische Ausdrücke in einzelne Terme auf
	\item Durchsuche resultierende NNF(W) nach Widersprüchen
\end{itemize}
\begin{block}{Beispiel}
Gegeben: Wissensbasis NNF(W): $ C(a) $  $ (\neg C \sqcap D)(a) $
\begin{itemize}
	\item Auflösen in Terme: $ (\neg C \sqcap D)(a) $ wird zu $ \neg C(a) \sqcap D(a) $
	\item Widerspruch in NNF(W) bei $ C(a) $ $ \neg C(a) $ 
\end{itemize}
\end{block}	
\end{frame}

\section{Fazit}
\begin{frame}{Fazit}
\end{frame}

\end{document}
