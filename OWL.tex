\documentclass{beamer}
\usepackage[utf8x]{inputenc}
\usepackage{color}
\usetheme{Boadilla}
\definecolor{specialblue}{rgb}{0.2,0.2,0.7}

\title[OWL]{Ontologien und formale Semantik in OWL}
\author{Timon Link, Matthias Jurisch}
\date{4. Dezember 2012}
\begin{document}
\begin{frame}
\titlepage
\end{frame}

\begin{frame}{Übersicht}
\end{frame}

%%%%%%%%%%%%%%%%%%%%%% Grundlagen %%%%%%%%%%%%%%%%%%%%%%%%%%
%%%%%%%%%%%%%%%%%%%%%% T W3C-Scheiss %%%%%%%%%%%%%%%%%%%%%%%%%%
\begin{frame}{W3C}
\end{frame}
%%%%%%%%%%%%%%%%%%%%%% T Rückblick RDF %%%%%%%%%%%%%%%%%%%%%%%
\begin{frame}{Rückblick RDF}
\end{frame}
%%%%%%%%%%%%%%%%%%%%%% M Ausdruck von RDF-Konstrukten einfacher mit 
%OWL (pointer auf Header auch)
%%%%%%%%%%%%%%%%%%%%%% M was kann nicht in RDF ausgedrückt werden?

%%%%%%%%%%%%%%%%%%%%%%-Inhalt-

%%%%%%%%%%%%%%%%%%%%%% T Neues in OWL, dass Ausdrucksmöglichkeiten von 
%RDF übersteigt (Hier Header beschreiben)
\begin{frame}{Neuerungen in OWL}

\end{frame}

\begin{frame}{Ontologie Kopf}

\end{frame}


%%%%%%%%%%%%%%%%%%%%%% M Probleme in OWL: Entscheidbarkeit etc.

%%%%%%%%%%%%%%%%%%%%%% -> Drei OWL-Varianten
%%%%%%%%%%%%%%%%%%%%%% T eingehena auf Tabelle 
\begin{frame}{OWL - Teilsprachen}
\begin{itemize}
	\item OWL ist durch Mächtigkeit sehr komplex
	\item Deshalb Aufteilung in drei Teilsprachen:
\begin{itemize}
	\item OWL Full
	\item OWL DL
	\item OWL LITE
\end{itemize}
	\item unterschiedliche Komplexität und Mächtigkeiten
\end{itemize}
\begin{block}{Relationen}
	Es gilt:\\
	$OWL Lite \subseteq OWL DL \subseteq OWL Full$
\end{block}
\end{frame}

%
\begin{frame}{OWL Full}
\begin{itemize}
	\item enthält als einzige Teilsprache \alert{ganz} RDFS
	\item enthält OWL DL und OWL LITE als Teilsprachen
	\item sehr ausdrucksstark
	\item ABER: \alert{unentscheidbar}
	\item deshalb keine Praxisrelevanz
\end{itemize}
\end{frame}

%
\begin{frame}{OWL DL}
\begin{itemize}
	\item enthält OWL Lite und ist Teilsprache von OWL Full
	\item enscheidbar
	\item sehr Praxisrelevant
	\item daher wird es von den meisten Softwarewerkzeugen unterstützt
\end{itemize}
\begin{block}{Komplexität}
	NExpTime
\end{block}
\end{frame}

%
\begin{frame}{OWL Lite}
\begin{itemize}
	\item ist Teilsprache von OWL DL und OWL Full
	\item entscheidbar
	\item weniger ausdrucksstark
	\item deswegen kaum Praxisrelevanz
\end{itemize}
\begin{block}{Komplexität}
	ExpTime
\end{block}
\end{frame}

%%%%%%%%%%%%%%%%%%%%%% M einige Sprachkonstrukte 


%%% ENDE %%
\end{document}
