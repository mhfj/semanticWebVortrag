\documentclass{beamer}
\usepackage{listings}
\usepackage[utf8x]{inputenc}
\usepackage{color}
\lstset{
  language=xml,
  float=htbp,
        tabsize=4,
        linewidth=\linewidth,
        breaklines=true,
        breakatwhitespace=true,
        basicstyle=\scriptsize\ttfamily,
        numberfirstline=false,
        numberstyle=\scriptsize,
        stepnumber=1,
        numbersep=5pt,
        showspaces=false,
        showtabs=false,
        showstringspaces=false,
        showlines=false,
        extendedchars=true,
        identifierstyle=\bfseries,
        keywordstyle=\bfseries,
        commentstyle=\itshape,
        stringstyle=\ttfamily,
        flexiblecolumns=false,
        fontadjust=true,
        captionpos=b,
}

\usetheme{Boadilla}
\definecolor{specialblue}{rgb}{0.2,0.2,0.7}

\title[OWL]{Ontologien und formale Semantik in OWL}
\author{Timon Link, Matthias Jurisch}
\date{4. Dezember 2012}
\begin{document}
\begin{frame}
\titlepage
\end{frame}

\begin{frame}{Übersicht}
\tableofcontents
\end{frame}

%%%%%%%%%%%%%%%%%%%%%% Grundlagen %%%%%%%%%%%%%%%%%%%%%%%%%%
%%%%%%%%%%%%%%%%%%%%%% T W3C-Scheiss %%%%%%%%%%%%%%%%%%%%%%%%%%
\section{Historie und Herkunft}
\begin{frame}{W3C}
\end{frame}
%%%%%%%%%%%%%%%%%%%%%% T Rückblick RDF %%%%%%%%%%%%%%%%%%%%%%%

\section{Kurzer Rückblick auf RDF}

%%%%%%%%%%%%%%%%%%%%%% M Ausdruck von RDF-Konstrukten einfacher mit OWL (pointer auf Header auch)

\section{Vereinfachung von RDF-Ausdrücken mit OWL}
\begin{frame}[fragile]{Vereinfachung von RDF-Ausdrücken mit
OWL}
\begin{block}{Vereinfachung von Ausdrücken}
\begin{itemize}
\item RDF/RDFS lässt sich durch OWL einfacher und abgekürzt ausdrücken
\end{itemize}
\end{block}

\begin{block}{Daher}
\begin{itemize}
\item Kurze Auflistung von Abkürzungen.
\item kennenlernen der wesentlichen Sprachkonstrukt
\end{itemize}
\end{block}


\end{frame}
\begin{frame}[fragile]{Vereinfachung von RDF-Ausdrücken mit
OWL}{Klassen}
\begin{block}{Klassen in RDFS}
\begin{lstlisting}[lang="xml"]
<rdf:Description about="&ex;Klassenname">
  <rdf:type resource="&rdfs;Class"/>
</rdf:Description>
\end{lstlisting}
\end{block}
\begin{block}{OWL}
\begin{lstlisting}[lang="xml"]
<owl:Class about="&ex;Klassenname"/>
\end{lstlisting}
\end{block}

\end{frame}

\begin{frame}[fragile]{Vereinfachung von RDF-Ausdrücken mit
OWL}{Individuen}
\begin{block}{Individuen in RDFS}
\begin{lstlisting}[lang="xml"]
<rdf:Description about="&ex;individuum">
  <rdf:type resource="&ex;Klassenname"/>
</rdf:Description>
\end{lstlisting}
\end{block}
\begin{block}{OWL}
\begin{lstlisting}[lang="xml"]
<ex:individuum about="&ex;Klassenname"/>
\end{lstlisting}
\end{block}
\end{frame}

\begin{frame}[fragile]{Vereinfachung von RDF-Ausdrücken mit
OWL}{Unterklassen}
\begin{block}{Unterklassen in RDFS}
\begin{lstlisting}[lang="xml"]
<rdf:Description about="&ex;Klassenname">
  <rdf:type resource="&rdfs;Class"/>
  <rdfs:subClassOf rdf:resource="&ex;Oberklasse"/>
</rdf:Description>
\end{lstlisting}
\end{block}
\begin{block}{OWL}
\begin{lstlisting}[lang="xml"]
<owl:Class about="&ex;Klassenname">
  <rdfs:subClassOf rdf:resource="&ex;Oberklasse"/>
</owl:Class>
\end{lstlisting}
\end{block}

\end{frame}



%%%%%%%%%%%%%%%%%%%%%% M was kann nicht in RDF ausgedrückt werden?

\section{Grenzen von RDF}

\begin{frame}[fragile]{Grenzen von RDF}
\begin{block}{Negative Aussagen in RDF}
\begin{itemize}
\item Nicht Möglich
\item Auch nicht über Umwege
\end{itemize}
\end{block}
\begin{exampleblock}{Beispiel}
\begin{lstlisting}
ex:sebastian rdf:type ex:Nichtraucher.
ex:sebastian rdf:type ex:Raucher.
\end{lstlisting}
$\rightarrow$ Kein Widerspruch, auch nicht Möglich diesen auszudrücken
\end{exampleblock}
\end{frame}

%%%%%%%%%%%%%%%%%%%%%%-Inhalt-
%%%%%%%%%%%%%%%%%%%%%% -> Drei OWL-Varianten
%%%%%%%%%%%%%%%%%%%%%% T eingehena auf Tabelle 

\section{OWL-Varianten}
\begin{frame}{OWL - Teilsprachen}
\begin{itemize}
	\item OWL ist durch Mächtigkeit sehr komplex
	\item Deshalb Aufteilung in drei Teilsprachen:
\begin{itemize}
	\item OWL Full
	\item OWL DL
	\item OWL LITE
\end{itemize}
	\item unterschiedliche Komplexität und Mächtigkeiten
\end{itemize}
\begin{block}{Relationen}
	Es gilt:\\
	$OWL Lite \subseteq OWL DL \subseteq OWL Full$
\end{block}
\end{frame}

%
\begin{frame}{OWL Full}
\begin{itemize}
	\item enthält als einzige Teilsprache \alert{ganz} RDFS
	\item enthält OWL DL und OWL LITE als Teilsprachen
	\item sehr ausdrucksstark
	\item ABER: \alert{unentscheidbar}
	\item deshalb keine Praxisrelevanz
\end{itemize}
\end{frame}

%
\begin{frame}{OWL DL}
\begin{itemize}
	\item enthält OWL Lite und ist Teilsprache von OWL Full
	\item enscheidbar
	\item sehr Praxisrelevant
	\item daher wird es von den meisten Softwarewerkzeugen unterstützt
\end{itemize}
\begin{block}{Komplexität}
	NExpTime
\end{block}
\end{frame}

%
\begin{frame}{OWL Lite}
\begin{itemize}
	\item ist Teilsprache von OWL DL und OWL Full
	\item entscheidbar
	\item weniger ausdrucksstark
	\item deswegen kaum Praxisrelevanz
\end{itemize}
\begin{block}{Komplexität}
	ExpTime
\end{block}
\end{frame}


%%%%%%%%%%%%%%%%%%%%%% T Neues in OWL, dass Ausdrucksmöglichkeiten von 
%RDF übersteigt (Hier Header beschreiben)
\section{Neue Ausdrucksmöglichkeiten in OWL}
\begin{frame}{Neuerungen in OWL}

\end{frame}

\begin{frame}{Kopf einer OWL-Ontologie}

\end{frame}



%%%%%%%%%%%%%%%%%%%%%% M einige Sprachkonstrukte 

\section{Fazit}
\begin{frame}{Fazit}
\end{frame}

\end{document}
