\documentclass{beamer}
\usepackage{listings}
\usepackage[utf8x]{inputenc}
\usepackage{color}
\lstset{
  language=xml,
  float=htbp,
        tabsize=4,
        linewidth=\linewidth,
        breaklines=true,
        breakatwhitespace=true,
        basicstyle=\scriptsize\ttfamily,
        numbers=left,
        numberfirstline=false,
        numberstyle=\scriptsize,
        stepnumber=1,
        numbersep=5pt,
        showspaces=false,
        showtabs=false,
        showstringspaces=false,
        showlines=false,
        extendedchars=true,
        identifierstyle=\bfseries,
        keywordstyle=\bfseries,
        commentstyle=\itshape,
        stringstyle=\ttfamily,
        flexiblecolumns=false,
        fontadjust=true,
        frame=trbl,
        captionpos=b,
        aboveskip=25pt,
}


\title[OWL]{Ontologien und formale Semantik in OWL}
\author{Timon Link, Matthias Jurisch}
\date{4. Dezember 2012}
\begin{document}
\begin{frame}
\titlepage
\end{frame}

\begin{frame}{Übersicht}
\tableofcontents
\end{frame}

%%%%%%%%%%%%%%%%%%%%%% Grundlagen %%%%%%%%%%%%%%%%%%%%%%%%%%
%%%%%%%%%%%%%%%%%%%%%% T W3C-Scheiss %%%%%%%%%%%%%%%%%%%%%%%%%%
\begin{frame}{W3C}
\end{frame}

%%%%%%%%%%%%%%%%%%%%%% T Rückblick RDF %%%%%%%%%%%%%%%%%%%%%%%
%%%%%%%%%%%%%%%%%%%%%% M Ausdruck von RDF-Konstrukten einfacher mit OWL (pointer auf Header auch)

\section{Vereinfachung von RDF-Ausdrücken mit OWL}
\begin{frame}[fragile]{Vereinfachung von RDF-Ausdrücken mit
OWL}
\begin{block}{Vereinfachung von Ausdrücken}
\begin{itemize}
\item RDF/RDFS lässt sich durch OWL einfacher und abgekürzt ausdrücken
\end{itemize}
\end{block}

\begin{block}{Daher}
\begin{itemize}
\item Kurze Auflistung von Abkürzungen.
\item kennenlernen der wesentlichen Sprachkonstrukt
\end{itemize}
\end{block}


\end{frame}
\begin{frame}[fragile]{Vereinfachung von RDF-Ausdrücken mit
OWL}{Klassen}
\begin{block}{Klassen in RDFS}
\begin{lstlisting}[lang="xml"]
<rdf:Description about="&ex;Klassenname">
  <rdf:type resource="&rdfs;Class"/>
</rdf:Description>
\end{lstlisting}
\end{block}
\begin{block}{OWL}
\begin{lstlisting}[lang="xml"]
<owl:Class about="&ex;Klassenname"/>
\end{lstlisting}
\end{block}

\end{frame}

\begin{frame}[fragile]{Vereinfachung von RDF-Ausdrücken mit
OWL}{Individuen}
\begin{block}{Individuen in RDFS}
\begin{lstlisting}[lang="xml"]
<rdf:Description about="&ex;individuum">
  <rdf:type resource="&ex;Klassenname"/>
</rdf:Description>
\end{lstlisting}
\end{block}
\begin{block}{OWL}
\begin{lstlisting}[lang="xml"]
<ex:individuum about="&ex;Klassenname"/>
\end{lstlisting}
\end{block}
\end{frame}

\begin{frame}[fragile]{Vereinfachung von RDF-Ausdrücken mit
OWL}{Unterklassen}
\begin{block}{Unterklassen in RDFS}
\begin{lstlisting}[lang="xml"]
<rdf:Description about="&ex;Klassenname">
  <rdf:type resource="&rdfs;Class"/>
  <rdfs:subClassOf rdf:resource="&ex;Oberklasse"/>
</rdf:Description>
\end{lstlisting}
\end{block}
\begin{block}{OWL}
\begin{lstlisting}[lang="xml"]
<owl:Class about="&ex;Klassenname">
  <rdfs:subClassOf rdf:resource="&ex;Oberklasse"/>
</owl:Class>
\end{lstlisting}
\end{block}

\end{frame}



%%%%%%%%%%%%%%%%%%%%%% M was kann nicht in RDF ausgedrückt werden?

\section{Grenzen von RDF}

\begin{frame}[fragile]{Grenzen von RDF}
\begin{block}{Negative Aussagen in RDF}
\begin{itemize}
\item Nicht Möglich
\item Auch nicht über Umwege
\end{itemize}
\end{block}
\begin{exampleblock}{Beispiel}
\begin{lstlisting}
ex:sebastian rdf:type ex:Nichtraucher.
ex:sebastian rdf:type ex:Raucher.
\end{lstlisting}
$\rightarrow$ Kein Widerspruch, auch nicht Möglich diesen auszudrücken
\end{exampleblock}
\end{frame}

%%%%%%%%%%%%%%%%%%%%%%-Inhalt-

%%%%%%%%%%%%%%%%%%%%%% T Neues in OWL, dass Ausdrucksmöglichkeiten von RDF übersteigt (Hier Header beschreiben)

%%%%%%%%%%%%%%%%%%%%%% M Probleme in OWL: Entscheidbarkeit etc.

\section{Probleme von OWL}

\begin{frame}{Probleme von OWL}
\begin{block}{Mächtigkeit}
\begin{itemize}
\item Ausdrucksmöglichkeiten mit OWL (fast) unbegrenzt
\item In gezeigter Form sehr Mächtig
\end{itemize}
\end{block}

\begin{alertblock}{Wo ist der Haken?}
\begin{itemize}
\item OWL in dieser Form nicht entscheidbar
\item Schlussfolgerungen liegen in \emph{EXP}
\end{itemize}
\end{alertblock}
\begin{block}{Lösung: Aufteilung in Varianten}
\begin{enumerate}
\item OWL Full
\item OWL DL
\item OWL Lite
\end{enumerate}
\end{block}
\end{frame}

%%%%%%%%%%%%%%%%%%%%%% -> Drei OWL-Varianten
%%%%%%%%%%%%%%%%%%%%%% T eingehena auf Tabelle 
%%%%%%%%%%%%%%%%%%%%%% M einige Sprachkonstrukte 


\end{document}
