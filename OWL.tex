\documentclass{beamer}
\usepackage[utf8x]{inputenc}
\usepackage{color}
\definecolor{specialblue}{rgb}{0.2,0.2,0.7}

\title[OWL]{Ontologien und formale Semantik in OWL}
\author{Timon Link, Matthias Jurisch}
\date{4. Dezember 2012}
\begin{document}
\begin{frame}
\titlepage
\end{frame}

\begin{frame}{Übersicht}
\end{frame}

%%%%%%%%%%%%%%%%%%%%%% Grundlagen %%%%%%%%%%%%%%%%%%%%%%%%%%
%%%%%%%%%%%%%%%%%%%%%% T W3C-Scheiss %%%%%%%%%%%%%%%%%%%%%%%%%%
\begin{frame}{W3C}
\end{frame}

%%%%%%%%%%%%%%%%%%%%%% T Rückblick RDF %%%%%%%%%%%%%%%%%%%%%%%
%%%%%%%%%%%%%%%%%%%%%% M Ausdruck von RDF-Konstrukten einfacher mit OWL (pointer auf Header auch)
%%%%%%%%%%%%%%%%%%%%%% M was kann nicht in RDF ausgedrückt werden?

%%%%%%%%%%%%%%%%%%%%%%-Inhalt-

%%%%%%%%%%%%%%%%%%%%%% T Neues in OWL, dass Ausdrucksmöglichkeiten von RDF übersteigt (Hier Header beschreiben)

%%%%%%%%%%%%%%%%%%%%%% M Probleme in OWL: Entscheidbarkeit etc.

%%%%%%%%%%%%%%%%%%%%%% -> Drei OWL-Varianten
%%%%%%%%%%%%%%%%%%%%%% T eingehena auf Tabelle 
%%%%%%%%%%%%%%%%%%%%%% M einige Sprachkonstrukte 


%%% ENDE %%
\end{document}
