\documentclass{beamer}
\usepackage{listings}
\usepackage[utf8x]{inputenc}
\usepackage{color}
\usetheme{Boadilla}
\definecolor{specialblue}{rgb}{0.2,0.2,0.7}

\title[OWL]{Ontologien und formale Semantik in OWL}
\author{Timon Link, Matthias Jurisch}
\date{4. Dezember 2012}
\begin{document}
\begin{frame}
\titlepage
\end{frame}

\begin{frame}{Übersicht}
\end{frame}

%%%%%%%%%%%%%%%%%%%%%% Grundlagen %%%%%%%%%%%%%%%%%%%%%%%%%%
%%%%%%%%%%%%%%%%%%%%%% T W3C-Scheiss %%%%%%%%%%%%%%%%%%%%%%%%%%
\begin{frame}{W3C}
\end{frame}

%%%%%%%%%%%%%%%%%%%%%% T Rückblick RDF %%%%%%%%%%%%%%%%%%%%%%%
%%%%%%%%%%%%%%%%%%%%%% M Ausdruck von RDF-Konstrukten einfacher mit OWL (pointer auf Header auch)

\begin{frame}[fragile]{Vereinfachung von RDF-Ausdrücken mit
OWL}
\begin{block}{Klassen in RDFS}
\begin{lstlisting}[lang="xml"]
<rdf:Description about="&ex;Klassennahme">
  <rdf:type resource="&rdfs;Class"/>
</rdf:Description>
\end{lstlisting}
\end{block}
\begin{block}{OWL}
\begin{lstlisting}[lang="xml"]
<owl:Class about="&ex;Klassennahme"/>
\end{lstlisting}
\end{block}

\end{frame}

%%%%%%%%%%%%%%%%%%%%%% M was kann nicht in RDF ausgedrückt werden?

%%%%%%%%%%%%%%%%%%%%%%-Inhalt-

%%%%%%%%%%%%%%%%%%%%%% T Neues in OWL, dass Ausdrucksmöglichkeiten von RDF übersteigt (Hier Header beschreiben)

%%%%%%%%%%%%%%%%%%%%%% M Probleme in OWL: Entscheidbarkeit etc.

%%%%%%%%%%%%%%%%%%%%%% -> Drei OWL-Varianten
%%%%%%%%%%%%%%%%%%%%%% T eingehena auf Tabelle 
%%%%%%%%%%%%%%%%%%%%%% M einige Sprachkonstrukte 


\end{document}
