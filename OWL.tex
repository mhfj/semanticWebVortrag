\documentclass{beamer}
\usepackage{listings}
\usepackage[utf8x]{inputenc}
\usepackage{color}
\lstset{
  language=xml,
  float=htbp,
        tabsize=4,
        linewidth=\linewidth,
        breaklines=true,
        breakatwhitespace=true,
        basicstyle=\scriptsize\ttfamily,
        numberfirstline=false,
        numberstyle=\scriptsize,
        stepnumber=1,
        numbersep=5pt,
        showspaces=false,
        showtabs=false,
        showstringspaces=false,
        showlines=false,
        extendedchars=true,
        identifierstyle=\bfseries,
        keywordstyle=\bfseries,
        commentstyle=\itshape,
        stringstyle=\ttfamily,
        flexiblecolumns=false,
        fontadjust=true,
        captionpos=b,
}

\usetheme{Boadilla}
\definecolor{specialblue}{rgb}{0.2,0.2,0.7}

\title[OWL]{Ontologien und formale Semantik in OWL}
\author{Timon Link, Matthias Jurisch}
\date{4. Dezember 2012}
\begin{document}
\begin{frame}
\titlepage
\end{frame}

\begin{frame}{Übersicht}
\tableofcontents
\end{frame}

%%%%%%%%%%%%%%%%%%%%%% Grundlagen %%%%%%%%%%%%%%%%%%%%%%%%%%
%%%%%%%%%%%%%%%%%%%%%% T W3C-Scheiss %%%%%%%%%%%%%%%%%%%%%%%%%%
\section{Historie und Herkunft}
\begin{frame}{Historie und Herkunft}
	\begin{itemize}
	\item OWL - Web Ontology Language
	\item 19. August 2003 als Vorschlag vorgestellt
	\end{itemize}
\end{frame}
%%%%%%%%%%%%%%%%%%%%%% T Rückblick RDF %%%%%%%%%%%%%%%%%%%%%%%

\section{Kurzer Rückblick auf RDF}
\begin{frame}{RDF - ein kurzer Rückblick}
\begin{itemize}
	\item \alert{R}esource \alert{D}escription \alert{L}anguage
	\item formale Sprache zur Beschreibungen einfacher Ontologien
	\item Graphen statt Bäume (trotz XML Syntax)
	\item Beschreibung eines Graphen mit Tripeln: (subject, predicat, object)
	\item Was hat RDF mit OWL zu tun?
\end{itemize}
\begin{block}{Genau das:}
	RDF(S) ist in OWL (Full) enthalten\\
	Daher ist jedes OWL Dokument auch ein RDF(S) Dokument.
\end{block}
\end{frame}

%%%%%%%%%%%%%%%%%%%%%% M Ausdruck von RDF-Konstrukten einfacher mit OWL (pointer auf Header auch)

\section{Vereinfachung von RDF-Ausdrücken mit OWL}
\begin{frame}[fragile]{Vereinfachung von RDF-Ausdrücken mit
OWL}
\begin{itemize}
\item RDF/RDFS lässt sich durch OWL einfacher und abgekürzt ausdrücken
\end{itemize}

\begin{block}{Daher}
\begin{itemize}
\item Kurze Auflistung von Abkürzungen.
\item kennenlernen der wesentlichen Sprachkonstrukte
\end{itemize}
\end{block}


\end{frame}
\begin{frame}[fragile]{Vereinfachung von RDF-Ausdrücken mit
OWL}{Klassen}
\begin{block}{Klassen in RDFS}
\begin{lstlisting}[lang="xml"]
<rdf:Description about="&ex;Klassenname">
  <rdf:type resource="&rdfs;Class"/>
</rdf:Description>
\end{lstlisting}
\end{block}
\begin{block}{OWL}
\begin{lstlisting}[lang="xml"]
<owl:Class about="&ex;Klassenname"/>
\end{lstlisting}
\end{block}

\end{frame}

\begin{frame}[fragile]{Vereinfachung von RDF-Ausdrücken mit
OWL}{Individuen}
\begin{block}{Individuen in RDFS}
\begin{lstlisting}[lang="xml"]
<rdf:Description about="&ex;individuum">
  <rdf:type resource="&ex;Klassenname"/>
</rdf:Description>
\end{lstlisting}
\end{block}
\begin{block}{OWL}
\begin{lstlisting}[lang="xml"]
<ex:individuum about="&ex;Klassenname"/>
\end{lstlisting}
\end{block}
\end{frame}

\begin{frame}[fragile]{Vereinfachung von RDF-Ausdrücken mit
OWL}{Unterklassen}
\begin{block}{Unterklassen in RDFS}
\begin{lstlisting}[lang="xml"]
<rdf:Description about="&ex;Klassenname">
  <rdf:type resource="&rdfs;Class"/>
  <rdfs:subClassOf rdf:resource="&ex;Oberklasse"/>
</rdf:Description>
\end{lstlisting}
\end{block}
\begin{block}{OWL}
\begin{lstlisting}[lang="xml"]
<owl:Class about="&ex;Klassenname">
  <rdfs:subClassOf rdf:resource="&ex;Oberklasse"/>
</owl:Class>
\end{lstlisting}
\end{block}

\end{frame}

\begin{frame}[fragile]{Vereinfachung von RDF-Ausdrücken mit OWL}{Properties}
\begin{itemize}
\item RDF-Propertys $\rightarrow$ OWL-Rollen
\begin{block}{Abstrakte Rollen}
\begin{itemize}
\item Verbindungen zwischen Individuen und Individuen
\item Knotentyp: \tt owl:ObjectProperty
\end{itemize}
\end{block}

\begin{block}{Konkrete Rollen}
\begin{itemize}
\item Verbindungen zwischen Individuen und Datenwerte
\item Knotentyp: \tt owl:DatatypeProperty
\end{itemize}
\end{block}

\item Beide sind Unterklasse von \texttt{rdf:Property}
\item Verwendung dementsprechend analog zu \texttt{rdf:Property}

\end{itemize}
\end{frame}


%%%%%%%%%%%%%%%%%%%%%% M was kann nicht in RDF ausgedrückt werden?

\section{Grenzen von RDF}

\begin{frame}[fragile]{Grenzen von RDF}
\begin{itemize}
\item Negative Aussagen in RDF
\begin{itemize}
\item Nicht Möglich
\item Auch nicht über Umwege
\end{itemize}
\end{itemize}
\begin{exampleblock}{Beispiel}
\begin{lstlisting}
ex:sebastian rdf:type ex:Nichtraucher.
ex:sebastian rdf:type ex:Raucher.
\end{lstlisting}
$\rightarrow$ Kein Widerspruch, auch nicht Möglich diesen auszudrücken
\end{exampleblock}

\begin{itemize}
\item OWL:
\begin{itemize}
\item erlaubt solche Ausdrücke
\item übersteigt damit die Mächtigkeit von RDF
\end{itemize}
\end{itemize}
\end{frame}

%%%%%%%%%%%%%%%%%%%%%%-Inhalt-
%%%%%%%%%%%%%%%%%%%%%% -> Drei OWL-Varianten
%%%%%%%%%%%%%%%%%%%%%% T eingehena auf Tabelle 

\section{OWL-Varianten}
\begin{frame}{OWL - Teilsprachen}
\begin{itemize}
	\item OWL ist durch Mächtigkeit sehr komplex
	\item Deshalb Aufteilung in drei Teilsprachen:
\begin{itemize}
	\item OWL Full
	\item OWL DL
	\item OWL LITE
\end{itemize}
	\item unterschiedliche Komplexität und Mächtigkeiten
\end{itemize}
\begin{block}{Relationen}
	Es gilt:\\
	$OWL Lite \subseteq OWL DL \subseteq OWL Full$
\end{block}
\end{frame}

%
\begin{frame}{OWL Full}
\begin{itemize}
	\item enthält als einzige Teilsprache \alert{ganz} RDFS
	\item enthält OWL DL und OWL LITE als Teilsprachen
	\item sehr ausdrucksstark
	\item ABER: \alert{unentscheidbar}
	\item deshalb keine Praxisrelevanz
\end{itemize}
\end{frame}

%
\begin{frame}{OWL DL}
\begin{itemize}
	\item enthält OWL Lite und ist Teilsprache von OWL Full
	\item enscheidbar
	\item sehr Praxisrelevant
	\item daher wird es von den meisten Softwarewerkzeugen unterstützt
\end{itemize}
\begin{block}{Komplexität}
	NExpTime
\end{block}
\end{frame}

%
\begin{frame}{OWL Lite}
\begin{itemize}
	\item ist Teilsprache von OWL DL und OWL Full
	\item entscheidbar
	\item weniger ausdrucksstark
	\item deswegen kaum Praxisrelevanz
\end{itemize}
\begin{block}{Komplexität}
	ExpTime
\end{block}
\end{frame}


%%%%%%%%%%%%%%%%%%%%%% T Neues in OWL, dass Ausdrucksmöglichkeiten von 
%RDF übersteigt (Hier Header beschreiben)
\section{Neue Ausdrucksmöglichkeiten in OWL}
\begin{frame}{Neuerungen in OWL}

\end{frame}

\begin{frame}[fragile]{Kopf einer OWL-Ontologie}
\begin{block}{Kopf einer OWL-Ontologie, Aufgabe:}
Speicherung von Informationen, die für das gesamte Dokument gültig sind.
\end{block}
\begin{itemize}
	\item Deklarationen von Namensräumen analog zu RDF(s)
	\item weitere Informationen werden in diesem  Tag definiert: 
\end{itemize}
\begin{block}{OWL Ontology}
\begin{lstlisting}[lang="xml"]
<owl:Ontology rdf:about=""> ... <owl:Ontology>
\end{lstlisting}
\end{block}
\end{frame}

\begin{frame}{Kopf einer OWL-Ontologie II}
Inhalte des OWL Headers:
\begin{itemize}
	\item bekannte Strukturen aus RDFS:
	\begin{itemize}
		\item  \emph{rdfs:comment} - Kommentare zur Ontologie
		\item  \emph{rdfs:label} - Beschreibung der Ontologie
		\item  \emph{rdfs:seeAlso} - Externe Verweise
		\item  \emph{rdfs:isDefinedBy} - Subjekt "wird definiert von" Objekt
	\end{itemize}
\end{itemize}
\begin{itemize}
	\item Neue Strukturen aus OWL:
	\begin{itemize}
		\item \emph{owl:versionInfo} - Versionsangabe der Ontologie
		\item \emph{owl:priorVersion} - Vorgängerversion
		\item \emph{owl:backwardCompatibleWith} - Abwärtskompatibilität
		\item \emph{owl:incompatibleWith} - Inkompatibel
		\item \emph{owl:DeprecatedClass} - veraltete Klasse
		\item \emph{owl:DeprecatedProperty} - veraltete Property
	\end{itemize}
\end{itemize}
\end{frame}

\begin{frame}[fragile]{Kopf einer Ontologie Beispiel}
\begin{block}{Beispiel}
\begin{lstlisting}[lang="xml"]
<owl:Ontology rdf:about=""> 
  <rdfs:comment rdf:datatype="string">
    Ontologie der Kochplattform 
    Rezepte zum Ausprobieren.		
  </rdfs:comment>
  <rdfs:label rdf:datatype="string">
    Rezepte, Kochen, Backen
  </rdfs:label>
  <owl:versionInfo>1.0.4</owl:versionInfo>
  <owl:DeprecatedClass rdf:ID="Suppen">
    <rdfs:comment  rdf:datatype="string">
      Eintopf ersetzt Suppe
    </rdfs:comment>
  </owl:DeprecatedClass>
<owl:Ontology>
\end{lstlisting}
\end{block}
\end{frame}


%%%%%%%%%%%%%%%%%%%%%% M einige Sprachkonstrukte 

\begin{frame}[fragile]{Beziehungen zwischen Individuen}
\begin{itemize}
\item In OWL lassen sich beziehungen zwischen Individuen ausdrücken
\item Es existiert keine \emph{Unique Name Assumption}
\item Mit \texttt{owl:sameAs} lässt sich ausdrücken, dass zwei
Individuen identisch sind.

\begin{exampleblock}{Beispiel}
\begin{lstlisting}[lang="xml"]
<Suppe rdf:about="Kohlsuppe"/>
<rdf:Description rdf:about="OmasKohlSuppe">
  <owl:sameAs rdf:resource="Kohlsuppe"/>
</rdf:Description>
\end{lstlisting}
\end{exampleblock}
\item \alert{Schlussfolgerung:} Im Beispiel ist \texttt{OmasKohlSuppe} auch eine
\texttt{Suppe}
\end{itemize}
\end{frame}

\begin{frame}[fragile]{Beziehungen zwischen Individuen}
\begin{itemize}
\item Ähnliches ist auch umgekehrt möglich
\item \texttt{owl:differentFrom} gibt dies an
\end{itemize}
\begin{block}{Abkürzende Schreibweise für viele Individuen}
\begin{lstlisting}[lang="xml"]
<owl:AllDifferent>
  <owl:distinctMembers rdf:parseType="Collection">
    <Suppe rdf:about="Gulaschsuppe"/>
    <Suppe rdf:about="UngarischeGulaschsuppe"/>
    <Suppe rdf:about="Kohlsuppe"/>
  </owl:distinctMembers>
</owl:Alldifferent>
\end{lstlisting}
\end{block}

\end{frame}
%
\begin{frame}[fragile]{Klassenbeziehungen}
\begin{itemize}
\item Äquivalente Relationen zu \texttt{owl:sameAs} und \texttt{owl:differentFrom}
existieren auch für Klassen
\item \texttt{owl:disjointWith}
\begin{itemize}
 \item disjunktheit zweier Klassen
\end{itemize}
\item \texttt{owl:equivalentClass}
\begin{itemize}
 \item Gleichheit zweier Klassen
\end{itemize}
\end{itemize}
\end{frame}
%
\begin{frame}[fragile]{Logische Konstruktoren}
\begin{itemize}
\item \emph{Disjunktion, Konjunktion} und \emph{Negation} können
ausgedrückt werden.
\item In \emph{OWL Lite} nur beschränkt erlaubt.
\item Werden als \emph{Logische Konstruktoren} bezeichnet
\item Auf diese Weise können komplexe Klassen erstellt werden
\item Sprachkonstrukte:
\begin{itemize}
\item \texttt{owl:intersectionOf}
\item \texttt{owl:unionOf}
\item \texttt{owl:complementOf}
\end{itemize}
\item Schachtelung Möglich
\end{itemize}
\end{frame}
%
\begin{frame}[fragile]{Logische Konstruktoren}
\begin{exampleblock}{Beispiel}
\begin{lstlisting}[lang="xml"]
<owl:Class rdf:About="Eintopf>
  <rdfs:subClassOf>
    <owl:unionOf rdf:parseType="Collection">
      <owl:intersectionOf rdf:parseType="Collection">
        <owl:Class rdf:about="Suppen"/>
        <owl:Class rdf:about="MachtSatt"/>
      </owl:intersectionOf>
      <owl:complementOf>
        <owl:Class rdf:about="KlareSuppen"/>
      </owl:complementOf>
    </owl:unionOf>
  </rdfs:subClassOf>
</owl:Class>
\end{lstlisting}
\end{exampleblock}

\end{frame}
%

\begin{frame}[fragile]{Rolleneigenschaften}{Restrictions}
\begin{itemize}
\item Rollen können eingeschränkt werden
\item Beschränkung ist von der Klasse abhängig
\item \texttt{owl:Restriction} beschreibt Klassen
\item \texttt{owl:onProperty} ist die Rolle, auf die sich bezogen wird
\item Mögliche Einschränkungen:
\begin{itemize}
\item Werte:
\begin{itemize}
\item \texttt{owl:allValuesFrom}
\item \texttt{owl:someValuesFrom}
\item \texttt{owl:hasValue}
\end{itemize}
\item Kardinalität:
\begin{itemize}
\item \texttt{owl:maxCardinality}
\item \texttt{owl:minCardinality}
\item \texttt{owl:Cardinality}
\end{itemize}

\end{itemize}

\end{itemize}


\end{frame}
%

\begin{frame}[fragile]{Rolleneigenschaften}{Restrictions}
\begin{exampleblock}{Beispiel}
\begin{lstlisting}[lang="xml"]
<owl:Class rdf:about="Gemueseeintopf">
  <rdfs:subClassOf>
    <owl:Restriction>
      <owl:onProperty rdf:resource="hatZutat"/>
      <owl:minCardinality rdf:datatype="&xsd;nonNegativeInteger">
        3
      </owl:minCardinality>
      <owl:someValuesFrom rdf:resource="Gemuese"/>
    </owl:Restriction>
  </rdfs:subClassOf>
</owl:Class>
\end{lstlisting}
\alert{Nicht:} \emph{,,Ein Gemüseeintopf hat mindestens drei
Zutaten vom Typ Gemüse.''}
\end{exampleblock}
\end{frame}


%
\begin{frame}[fragile]{Rolleneigenschaften}{Beziehungen}
\begin{itemize}
\item Spezifikation von sogenannten \emph{characteristics}
\item 
\end{itemize}
\end{frame}
%%%%%%%%%%%%%%%%%%%%%% formale Semantik von OWL

\section{Formale Semantik von OWL}

\begin{frame}{Formale Semantik von OWL}
\end{frame}


\section{Fazit}
\begin{frame}{Fazit}
\end{frame}

\end{document}
